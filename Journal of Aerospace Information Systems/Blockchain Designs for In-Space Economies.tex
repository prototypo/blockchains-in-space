% Blockchain Designs for In-Space Economies
%
% This paper uses the AIAA Basic template, which seems to suit our needs.
%
% `template_basic.tex' - A bare-bones example of using the AIAA class.
%                        For a more advanced usage, see `template_advanced.tex'.
%
% Typical processing for PostScript (PS) output:
%
%  latex template_basic
%  latex template_basic   (repeat as needed to resolve references)
%
%  xdvi template_basic    (onscreen draft display)
%  dvips template_basic   (postscript)
%  gv template_basic.ps   (onscreen display)
%  lpr template_basic.ps  (hardcopy)
%
% With the above, only Encapsulated PostScript (EPS) images can be used.
%
% Typical processing for Portable Document Format (PDF) output:
%
%  pdflatex template_basic
%  pdflatex template_basic      (repeat as needed to resolve references)
%
%  acroread template_basic.pdf  (onscreen display)
%
% If you have EPS figures, you will need to use the epstopdf script
% to convert them to PDF because PDF is a limmited subset of EPS.
% pdflatex accepts a variety of other image formats such as JPG, TIF,
% PNG, and so forth -- check the documentation for your version.
%
% If you do *not* specify suffixes when using the graphicx package's
% \includegraphics command, latex and pdflatex will automatically select
% the appropriate figure format from those available.  This allows you
% to produce PS and PDF output from the same LaTeX source file.
%
% To generate a large format (e.g., 11"x17") PostScript copy for editing
% purposes, use
%
%  dvips -x 1467 -O -0.65in,0.85in -t tabloid template_basic
%
% For further details and support, read the Users Manual, aiaa.pdf.
%
% This software is released under the terms of the LaTeX Project Public
% License.  Copyright (C) 2004 by Bil Kleb, Bill Wood, and Erich Knausenberger.


\documentclass[]{aiaa-tc}% insert '[draft]' option to show overfull boxes

% Author-specific packages
\usepackage[breaklinks=true]{hyperref}
\hypersetup{colorlinks, citecolor=green, filecolor=black, linkcolor=blue, urlcolor=blue }
% End author-specific packages

 \title{Blockchain Designs for In-Space Economies}

% TODO: Not ordered yet
 \author{
  David Hyland-Wood%
    \thanks{PegaSys, ConsenSys Australia, Brisbane, Queensland, Australia}
    \thanks{School of Information Technology and Electrical Engineering, The University of Queensland, Brisbane, Queensland, Australia}\\
  \and Peter Robinson\thanksibid{1} \thanksibid{2}
  \and Brett Henderson\thanksibid{1}
  \and Christopher Hare\thanksibid{1}
  \and Roberto Saltini\thanksibid{1}
  \and Sandra Johnson\thanksibid{1}
  \and
  Chris Lewicki%
   \thanks{ConsenSys Space, Redmond, Washington, USA, AIAA Member.}
 }

 % Data used by 'handcarry' option if invoked
 \AIAApapernumber{YEAR-NUMBER}
 \AIAAconference{Conference Name, Date, and Location}
 \AIAAcopyright{\AIAAcopyrightD{YEAR}}

 % Define commands to assure consistent treatment throughout document
 \newcommand{\eqnref}[1]{(\ref{#1})}
 \newcommand{\class}[1]{\texttt{#1}}
 \newcommand{\package}[1]{\texttt{#1}}
 \newcommand{\file}[1]{\texttt{#1}}
 \newcommand{\BibTeX}{\textsc{Bib}\TeX}

\begin{document}

\maketitle

\begin{abstract}
TODO
\end{abstract}

\section*{Nomenclature}

\begin{tabbing}
  XXX \= \kill% this line sets tab stop
  $n$ \> A blockchain node \\
  $N_{n}$ \> The total number of participating blockchain nodes \\
  $v$ \> A validator participating in a blockchain consensus protocol \\
  $N_{v}$ \> The total number of participating validators \\
  \textit{Subscript}\\
  $i$ \> Variable number \\
 \end{tabbing}

%%%%%%%%
% SECTION %
%%%%%%%%
\section{Introduction}

TODO: Partially from \href{https://docs.google.com/document/d/1K9YtM1mFtg6TUXoUPOeFBk6uPVdNTkJD7JFjJ4v7Ing/edit}{Defining a Space Blockchain}

%%%%%%%%
% SECTION %
%%%%%%%%
\section{Literature Review}

TODO: Primarily from \href{https://docs.google.com/document/d/1rU15AehftYQ6U1uiUhi44GROBVatPzkukOgTyjBBE_M/edit}{Literature Review}

TODO: Remove this test citation\cite{wust_you_2017}.

%%%%%%%%
% SECTION %
%%%%%%%%
\section{Consensus Algorithm Selection}

TODO: Primarily from \href{https://docs.google.com/document/d/16O1zB_lcD3egKadkRsFOPac2bynLB9gTbNVr9kFf3HA/edit}{Consensus Algorithm for an In-Space Economy} and \href{https://docs.google.com/document/d/1rzcag5pdtEJBVtK_JrG3c8htZkn-xGRbSOg5MPRNL3s/edit}{Algorithm Evaluation for Blockchains in Space}


\section{Desired Blockchain Properties for Solar System Domains}

TODO: Blockchain properties for the three considered domains:
\begin{itemize}
\item Cislunar and cisdeimotic space
\item Interplanetary space
\item Local autonomy in the outer Solar System
\end{itemize}

\subsection{Near Planets}

TODO: Blockchain properties for operations with the orbits of an outer moon (e.g. cislunar and cisdeimotic space).


\subsection{Interplanetary Space}

TODO: Blockchain properties for interplanetary blockchain operations (e.g. Earth-Mars).


\subsection{Local Autonomy in the Outer Solar System}

TODO: Asynchronous protocol for lex mercatoria operations between spacecraft.


\section{Consensus Near Planets}

TODO: Details of a consensus algorithm for the cislunar  and cisdeimotic domains


\section{Results}

TODO


\section{Conclusion}

TODO


\section*{Acknowledgments}

TODO: Thank PegaSys, ConsenSys.


\bibliography{bibliography}
\bibliographystyle{aiaa}
%\begin{thebibliography}{9}% maximum number of references (for label width)
% \bibitem{rebek:82bk}
% Rebek, A., {\it Fickle Rocks}, Fink Publishing, Chesapeake, 1982.
%\end{thebibliography}

\end{document}

% $Id: template_basic.tex,v 1.5 2004/05/23 12:49:44 kleb Exp $
